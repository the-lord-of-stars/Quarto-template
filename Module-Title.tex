% Options for packages loaded elsewhere
\PassOptionsToPackage{unicode}{hyperref}
\PassOptionsToPackage{hyphens}{url}
%
\documentclass[
  letterpaper,
  oneside]{book}

\usepackage{amsmath,amssymb}
\usepackage{iftex}
\ifPDFTeX
  \usepackage[T1]{fontenc}
  \usepackage[utf8]{inputenc}
  \usepackage{textcomp} % provide euro and other symbols
\else % if luatex or xetex
  \usepackage{unicode-math}
  \defaultfontfeatures{Scale=MatchLowercase}
  \defaultfontfeatures[\rmfamily]{Ligatures=TeX,Scale=1}
\fi
\usepackage{lmodern}
\ifPDFTeX\else  
    % xetex/luatex font selection
\fi
% Use upquote if available, for straight quotes in verbatim environments
\IfFileExists{upquote.sty}{\usepackage{upquote}}{}
\IfFileExists{microtype.sty}{% use microtype if available
  \usepackage[]{microtype}
  \UseMicrotypeSet[protrusion]{basicmath} % disable protrusion for tt fonts
}{}
\makeatletter
\@ifundefined{KOMAClassName}{% if non-KOMA class
  \IfFileExists{parskip.sty}{%
    \usepackage{parskip}
  }{% else
    \setlength{\parindent}{0pt}
    \setlength{\parskip}{6pt plus 2pt minus 1pt}}
}{% if KOMA class
  \KOMAoptions{parskip=half}}
\makeatother
\usepackage{xcolor}
\setlength{\emergencystretch}{3em} % prevent overfull lines
\setcounter{secnumdepth}{3}
% Make \paragraph and \subparagraph free-standing
\makeatletter
\ifx\paragraph\undefined\else
  \let\oldparagraph\paragraph
  \renewcommand{\paragraph}{
    \@ifstar
      \xxxParagraphStar
      \xxxParagraphNoStar
  }
  \newcommand{\xxxParagraphStar}[1]{\oldparagraph*{#1}\mbox{}}
  \newcommand{\xxxParagraphNoStar}[1]{\oldparagraph{#1}\mbox{}}
\fi
\ifx\subparagraph\undefined\else
  \let\oldsubparagraph\subparagraph
  \renewcommand{\subparagraph}{
    \@ifstar
      \xxxSubParagraphStar
      \xxxSubParagraphNoStar
  }
  \newcommand{\xxxSubParagraphStar}[1]{\oldsubparagraph*{#1}\mbox{}}
  \newcommand{\xxxSubParagraphNoStar}[1]{\oldsubparagraph{#1}\mbox{}}
\fi
\makeatother


\providecommand{\tightlist}{%
  \setlength{\itemsep}{0pt}\setlength{\parskip}{0pt}}\usepackage{longtable,booktabs,array}
\usepackage{calc} % for calculating minipage widths
% Correct order of tables after \paragraph or \subparagraph
\usepackage{etoolbox}
\makeatletter
\patchcmd\longtable{\par}{\if@noskipsec\mbox{}\fi\par}{}{}
\makeatother
% Allow footnotes in longtable head/foot
\IfFileExists{footnotehyper.sty}{\usepackage{footnotehyper}}{\usepackage{footnote}}
\makesavenoteenv{longtable}
\usepackage{graphicx}
\makeatletter
\newsavebox\pandoc@box
\newcommand*\pandocbounded[1]{% scales image to fit in text height/width
  \sbox\pandoc@box{#1}%
  \Gscale@div\@tempa{\textheight}{\dimexpr\ht\pandoc@box+\dp\pandoc@box\relax}%
  \Gscale@div\@tempb{\linewidth}{\wd\pandoc@box}%
  \ifdim\@tempb\p@<\@tempa\p@\let\@tempa\@tempb\fi% select the smaller of both
  \ifdim\@tempa\p@<\p@\scalebox{\@tempa}{\usebox\pandoc@box}%
  \else\usebox{\pandoc@box}%
  \fi%
}
% Set default figure placement to htbp
\def\fps@figure{htbp}
\makeatother

\usepackage[a4paper, total={6in, 9in}]{geometry}

\usepackage{fancyhdr}
\pagestyle{fancy}
\renewcommand{\headrulewidth}{0pt}
\fancyhf{}
\fancyhead[L]{\nouppercase{\leftmark}}
\fancyhead[R]{\nouppercase{\rightmark}}
\fancyfoot[C]{\thepage}

\usepackage{titling}
\pretitle{\begin{center}
\includegraphics[width=1.8837in,height=0.7in]{logos/UoNTransparent.png}\LARGE\\}
\posttitle{\end{center}}

\usepackage{booktabs}
\usepackage{amsthm}
\usepackage{amsmath}
\usepackage{amssymb}

\makeatletter
\def\thm@space@setup{%
  \thm@preskip=8pt plus 2pt minus 4pt
  \thm@postskip=\thm@preskip
}
\makeatother

\numberwithin{equation}{section}
\numberwithin{figure}{section}

\usepackage[T1]{fontenc}                            % Font Styling
\usepackage{helvet,mathrsfs}
\renewcommand{\familydefault}{\sfdefault}

\usepackage{tikz}
\usetikzlibrary{arrows}


% ----------------------------------------------------------------------
%           User Created Environments
% ----------------------------------------------------------------------

%\theoremstyle{definition}

\newtheoremstyle{break}% name
  {}%         Space above, empty = `usual value'
  {}%         Space below
  {}% Body font
  {}%         Indent amount (empty = no indent, \parindent = para indent)
  {\bfseries}% Thm head font
  {}%        Punctuation after thm head
  {\newline}% Space after thm head: \newline = linebreak
  {}%         Thm head spec
\theoremstyle{break}

%%%-----Note-----%%%
\newtheorem*{note}{Note}
%%%-----Tip-----%%%
\newtheorem*{tip}{Tip}
%%%-----Discussion-----%%%
\newtheorem*{discussion}{Discussion}
%%%-----Activity-----%%%
\newtheorem*{activity}{Lecture Activity}
%%%-----Common Mistake-----%%%
\newtheorem*{mistake}{Common mistake}
%%%-----Key Point-----%%%
\newtheorem*{keypoint}{Key point}
%%%-----Notation-----%%%
\newtheorem*{notation}{Notation}

\makeatletter
\@ifpackageloaded{bookmark}{}{\usepackage{bookmark}}
\makeatother
\makeatletter
\@ifpackageloaded{caption}{}{\usepackage{caption}}
\AtBeginDocument{%
\ifdefined\contentsname
  \renewcommand*\contentsname{Table of contents}
\else
  \newcommand\contentsname{Table of contents}
\fi
\ifdefined\listfigurename
  \renewcommand*\listfigurename{List of Figures}
\else
  \newcommand\listfigurename{List of Figures}
\fi
\ifdefined\listtablename
  \renewcommand*\listtablename{List of Tables}
\else
  \newcommand\listtablename{List of Tables}
\fi
\ifdefined\figurename
  \renewcommand*\figurename{Figure}
\else
  \newcommand\figurename{Figure}
\fi
\ifdefined\tablename
  \renewcommand*\tablename{Table}
\else
  \newcommand\tablename{Table}
\fi
}
\@ifpackageloaded{float}{}{\usepackage{float}}
\floatstyle{ruled}
\@ifundefined{c@chapter}{\newfloat{codelisting}{h}{lop}}{\newfloat{codelisting}{h}{lop}[chapter]}
\floatname{codelisting}{Listing}
\newcommand*\listoflistings{\listof{codelisting}{List of Listings}}
\makeatother
\makeatletter
\makeatother
\makeatletter
\@ifpackageloaded{caption}{}{\usepackage{caption}}
\@ifpackageloaded{subcaption}{}{\usepackage{subcaption}}
\makeatother

\usepackage{bookmark}

\IfFileExists{xurl.sty}{\usepackage{xurl}}{} % add URL line breaks if available
\urlstyle{same} % disable monospaced font for URLs
\hypersetup{
  pdftitle={Module Title},
  pdfauthor={NAME},
  hidelinks,
  pdfcreator={LaTeX via pandoc}}


\title{Module Title}
\author{NAME}
\date{}

\begin{document}
\frontmatter
\maketitle

\renewcommand*\contentsname{Table of contents}
{
\setcounter{tocdepth}{2}
\tableofcontents
}
\listoffigures
\listoftables

\mainmatter
\bookmarksetup{startatroot}

\chapter*{Introduction to the module}\label{introduction-to-the-module}
\addcontentsline{toc}{chapter}{Introduction to the module}

\markboth{Introduction to the module}{Introduction to the module}

Enter your introduction to the module here

\section*{Learning objectives}\label{learning-objectives}
\addcontentsline{toc}{section}{Learning objectives}

\markright{Learning objectives}

Enter your learning objectives here

\section*{Convenor}\label{convenor}
\addcontentsline{toc}{section}{Convenor}

\markright{Convenor}

Enter your name and contact details here

\section*{Module resources}\label{module-resources}
\addcontentsline{toc}{section}{Module resources}

\markright{Module resources}

Enter any resources you would like to share with students

\section*{Module outline}\label{module-outline}
\addcontentsline{toc}{section}{Module outline}

\markright{Module outline}

Enter the timeline for the module here

\begin{longtable}[]{@{}lll@{}}
\toprule\noalign{}
Week & Topic & Assignment \\
\midrule\noalign{}
\endhead
\bottomrule\noalign{}
\endlastfoot
1 & & \\
2 & & \\
\ldots{} & \ldots{} & \ldots{} \\
\end{longtable}

\section*{Assessments}\label{assessments}
\addcontentsline{toc}{section}{Assessments}

\markright{Assessments}

Enter any assessments you would like to have for this module

\bookmarksetup{startatroot}

\chapter*{Feedback}\label{feedback}
\addcontentsline{toc}{chapter}{Feedback}

\markboth{Feedback}{Feedback}

Enter the feedback forum here:

\section*{Feedback Forum Link}\label{feedback-forum-link}
\addcontentsline{toc}{section}{Feedback Forum Link}

\markright{Feedback Forum Link}

Put the link for the feedback forum here:

\ldots{}

\part{Week 1: Your Title Here}

Enter your content here.

\chapter*{Lecture}\label{lecture}
\addcontentsline{toc}{chapter}{Lecture}

\markboth{Lecture}{Lecture}

\section*{PPT LINK}\label{ppt-link}
\addcontentsline{toc}{section}{PPT LINK}

\markright{PPT LINK}

Put the link to your PPT here.

\section*{Record}\label{record}
\addcontentsline{toc}{section}{Record}

\markright{Record}

Put your recording link here.

\section*{Resources}\label{resources}
\addcontentsline{toc}{section}{Resources}

\markright{Resources}

Put any resources you want to share with the class here.

\chapter*{Lab}\label{lab}
\addcontentsline{toc}{chapter}{Lab}

\markboth{Lab}{Lab}

\section*{Goal of the Lab}\label{goal-of-the-lab}
\addcontentsline{toc}{section}{Goal of the Lab}

\markright{Goal of the Lab}

Enter the goal of the lab here.

\section*{Plan}\label{plan}
\addcontentsline{toc}{section}{Plan}

\markright{Plan}

Enter the plan of the lab here.

\part{Week 2: Your Title Here}

Enter your content here.

\chapter*{Lecture}\label{lecture-1}
\addcontentsline{toc}{chapter}{Lecture}

\markboth{Lecture}{Lecture}

\section*{PPT LINK}\label{ppt-link-1}
\addcontentsline{toc}{section}{PPT LINK}

\markright{PPT LINK}

Put the link to your PPT here.

\section*{Record}\label{record-1}
\addcontentsline{toc}{section}{Record}

\markright{Record}

Put your recording link here.

\section*{Resources}\label{resources-1}
\addcontentsline{toc}{section}{Resources}

\markright{Resources}

Put any resources you want to share with the class here.

\chapter*{Lab}\label{lab-1}
\addcontentsline{toc}{chapter}{Lab}

\markboth{Lab}{Lab}

\section*{Goal of the Lab}\label{goal-of-the-lab-1}
\addcontentsline{toc}{section}{Goal of the Lab}

\markright{Goal of the Lab}

Enter the goal of the lab here.

\section*{Plan}\label{plan-1}
\addcontentsline{toc}{section}{Plan}

\markright{Plan}

Enter the plan of the lab here.

\part{Week 3: Your Title Here}

Enter your content here.

\chapter*{Lecture}\label{lecture-2}
\addcontentsline{toc}{chapter}{Lecture}

\markboth{Lecture}{Lecture}

\section*{PPT LINK}\label{ppt-link-2}
\addcontentsline{toc}{section}{PPT LINK}

\markright{PPT LINK}

Put the link to your PPT here.

\section*{Record}\label{record-2}
\addcontentsline{toc}{section}{Record}

\markright{Record}

Put your recording link here.

\section*{Resources}\label{resources-2}
\addcontentsline{toc}{section}{Resources}

\markright{Resources}

Put any resources you want to share with the class here.

\chapter*{Lab}\label{lab-2}
\addcontentsline{toc}{chapter}{Lab}

\markboth{Lab}{Lab}

\section*{Goal of the Lab}\label{goal-of-the-lab-2}
\addcontentsline{toc}{section}{Goal of the Lab}

\markright{Goal of the Lab}

Enter the goal of the lab here.

\section*{Plan}\label{plan-2}
\addcontentsline{toc}{section}{Plan}

\markright{Plan}

Enter the plan of the lab here.

\part{Week 4: Your Title Here}

Enter your content here.

\chapter*{Lecture}\label{lecture-3}
\addcontentsline{toc}{chapter}{Lecture}

\markboth{Lecture}{Lecture}

\section*{PPT LINK}\label{ppt-link-3}
\addcontentsline{toc}{section}{PPT LINK}

\markright{PPT LINK}

Put the link to your PPT here.

\section*{Record}\label{record-3}
\addcontentsline{toc}{section}{Record}

\markright{Record}

Put your recording link here.

\section*{Resources}\label{resources-3}
\addcontentsline{toc}{section}{Resources}

\markright{Resources}

Put any resources you want to share with the class here.

\chapter*{Lab}\label{lab-3}
\addcontentsline{toc}{chapter}{Lab}

\markboth{Lab}{Lab}

\section*{Goal of the Lab}\label{goal-of-the-lab-3}
\addcontentsline{toc}{section}{Goal of the Lab}

\markright{Goal of the Lab}

Enter the goal of the lab here.

\section*{Plan}\label{plan-3}
\addcontentsline{toc}{section}{Plan}

\markright{Plan}

Enter the plan of the lab here.

\part{Week 5: Your Title Here}

Enter your content here.

\chapter*{Lecture}\label{lecture-4}
\addcontentsline{toc}{chapter}{Lecture}

\markboth{Lecture}{Lecture}

\section*{PPT LINK}\label{ppt-link-4}
\addcontentsline{toc}{section}{PPT LINK}

\markright{PPT LINK}

Put the link to your PPT here.

\section*{Record}\label{record-4}
\addcontentsline{toc}{section}{Record}

\markright{Record}

Put your recording link here.

\section*{Resources}\label{resources-4}
\addcontentsline{toc}{section}{Resources}

\markright{Resources}

Put any resources you want to share with the class here.

\chapter*{Lab}\label{lab-4}
\addcontentsline{toc}{chapter}{Lab}

\markboth{Lab}{Lab}

\section*{Goal of the Lab}\label{goal-of-the-lab-4}
\addcontentsline{toc}{section}{Goal of the Lab}

\markright{Goal of the Lab}

Enter the goal of the lab here.

\section*{Plan}\label{plan-4}
\addcontentsline{toc}{section}{Plan}

\markright{Plan}

Enter the plan of the lab here.

\part{Week 6: Your Title Here}

Enter your content here.

\chapter*{Lecture}\label{lecture-5}
\addcontentsline{toc}{chapter}{Lecture}

\markboth{Lecture}{Lecture}

\section*{PPT LINK}\label{ppt-link-5}
\addcontentsline{toc}{section}{PPT LINK}

\markright{PPT LINK}

Put the link to your PPT here.

\section*{Record}\label{record-5}
\addcontentsline{toc}{section}{Record}

\markright{Record}

Put your recording link here.

\section*{Resources}\label{resources-5}
\addcontentsline{toc}{section}{Resources}

\markright{Resources}

Put any resources you want to share with the class here.

\chapter*{Lab}\label{lab-5}
\addcontentsline{toc}{chapter}{Lab}

\markboth{Lab}{Lab}

\section*{Goal of the Lab}\label{goal-of-the-lab-5}
\addcontentsline{toc}{section}{Goal of the Lab}

\markright{Goal of the Lab}

Enter the goal of the lab here.

\section*{Plan}\label{plan-5}
\addcontentsline{toc}{section}{Plan}

\markright{Plan}

Enter the plan of the lab here.

\part{Week 7: Your Title Here}

Enter your content here.

\chapter*{Lecture}\label{lecture-6}
\addcontentsline{toc}{chapter}{Lecture}

\markboth{Lecture}{Lecture}

\section*{PPT LINK}\label{ppt-link-6}
\addcontentsline{toc}{section}{PPT LINK}

\markright{PPT LINK}

Put the link to your PPT here.

\section*{Record}\label{record-6}
\addcontentsline{toc}{section}{Record}

\markright{Record}

Put your recording link here.

\section*{Resources}\label{resources-6}
\addcontentsline{toc}{section}{Resources}

\markright{Resources}

Put any resources you want to share with the class here.

\chapter*{Lab}\label{lab-6}
\addcontentsline{toc}{chapter}{Lab}

\markboth{Lab}{Lab}

\section*{Goal of the Lab}\label{goal-of-the-lab-6}
\addcontentsline{toc}{section}{Goal of the Lab}

\markright{Goal of the Lab}

Enter the goal of the lab here.

\section*{Plan}\label{plan-6}
\addcontentsline{toc}{section}{Plan}

\markright{Plan}

Enter the plan of the lab here.

\part{Week 8: Your Title Here}

Enter your content here.

\chapter*{Lecture}\label{lecture-7}
\addcontentsline{toc}{chapter}{Lecture}

\markboth{Lecture}{Lecture}

\section*{PPT LINK}\label{ppt-link-7}
\addcontentsline{toc}{section}{PPT LINK}

\markright{PPT LINK}

Put the link to your PPT here.

\section*{Record}\label{record-7}
\addcontentsline{toc}{section}{Record}

\markright{Record}

Put your recording link here.

\section*{Resources}\label{resources-7}
\addcontentsline{toc}{section}{Resources}

\markright{Resources}

Put any resources you want to share with the class here.

\chapter*{Lab}\label{lab-7}
\addcontentsline{toc}{chapter}{Lab}

\markboth{Lab}{Lab}

\section*{Goal of the Lab}\label{goal-of-the-lab-7}
\addcontentsline{toc}{section}{Goal of the Lab}

\markright{Goal of the Lab}

Enter the goal of the lab here.

\section*{Plan}\label{plan-7}
\addcontentsline{toc}{section}{Plan}

\markright{Plan}

Enter the plan of the lab here.

\bookmarksetup{startatroot}

\chapter*{\texorpdfstring{{Coursework
1}}{Coursework 1}}\label{coursework-1}
\addcontentsline{toc}{chapter}{{Coursework 1}}

\markboth{{Coursework 1}}{{Coursework 1}}

Enter your specification here.

\bookmarksetup{startatroot}

\chapter*{\texorpdfstring{{Coursework
2}}{Coursework 2}}\label{coursework-2}
\addcontentsline{toc}{chapter}{{Coursework 2}}

\markboth{{Coursework 2}}{{Coursework 2}}

Enter your specification here.

\cleardoublepage
\phantomsection
\addcontentsline{toc}{part}{Appendices}
\appendix

\chapter{Example}\label{example}

This is just an example.


\backmatter


\end{document}
